\documentclass[12pt]{MUNThesisBibTeX}
%set MUN Thesis Guidelines margins
\usepackage[left=3.8cm, right=2.5cm, top=3cm, bottom=3cm]{geometry}

%lists package and bullet formats
\usepackage{paralist}
\usepackage{bbding}
%enhanced graphics support
\usepackage{graphicx}
%formatting for websites and e-mail addresses
\usepackage{url}
%pageheaders and footers in LaTeX2e
\usepackage{fancyhdr}
%footnote options
\usepackage{footmisc}
%algorithms
\usepackage{algorithmic}
%source code printer
\usepackage{listings}
%read/write verbatim TeX code
\usepackage{fancyvrb}

%CJK commands
%Simplified Chinese
\newcommand{\zhs}[1]{\begin{CJK}{UTF8}{gbsn}#1\end{CJK}}
%Traditional Chinese
\newcommand{\zht}[1]{\begin{CJK}{UTF8}{bsmi}#1\end{CJK}}
%Japanese Kanji, Hiragana and Katakana
\newcommand{\nippon}[1]{\begin{CJK}{UTF8}{ipam}#1\end{CJK}}
%Korean Hangul
%\newfontfamily\hangul{Batang}
%Korean Hanja
\newcommand{\hanja}[1]{\begin{CJK}{UTF8}{bsmi}#1\end{CJK}}
 

%package conflicts - only one may be activated

%\usepackage{arabtex}
%\usepackage{devanagari}


%enable endnotes - if you wish to enable endnotes, remove the % sign at the start of the following line
%\usepackage{endnotes}
%\let\footnote=\endnote



%THIS IS WHERE YOU ENTER THE TITLE OF YOUR THESIS
\title{Title of the Master's/Doctoral Thesis}

%THIS IS WHERE YOU ENTER YOUR NAME
\author{Student's Name}

%THIS IS WHERE YOU ENTER THE NAME OF YOUR DEGREE
\deg{Degree Name}

%THIS IS WHERE YOU ENTER THE NAME OF YOUR DEPARTMENT, SCHOOL, or FACULTY
\fac{Department/School/Faculty Name}

%THIS IS WHERE YOU ENTER THE DATE YOU SUBMITTED YOUR THESIS OR DISSERTATION
\date{Month and Year Submitted}

%THIS IS WHERE YOU ENTER THE YEAR OF GRADUATION 
\copyrightyear{Year of graduation}


%no paragraph indentation
\setlength\parindent{0pt}
\newtheorem{theorem}{Theorem}[section]
\newtheorem{definition}{Definition}[section]
\newtheorem{lemma}{Lemma}[section]
\newtheorem{notation}{Notation}[section]
\begin{document}
\muntitlepage

%set the hierarchical drilldown to 3
\setcounter{secnumdepth}{3} \setcounter{tocdepth}{3}

%set pagination to Roman numerals and begin at page i
\pagenumbering{roman} \setcounter{page}{1}

\doublespacing
\setlength{\topmargin}{-.5in}

\chapter*{Abstract}
\addcontentsline{toc}{chapter}{Abstract}
%enter text for the abstract below
Text for the abstract.

\chapter*{Acknowledgements}
\addcontentsline{toc}{chapter}{Acknowledgments} 
%enter text for the acknowledgements below
Text for the acknowledgements.

%%-----------Table of Contents------------------
\renewcommand{\contentsname}{Table of Contents}
\tableofcontents{}
\addcontentsline{toc}{chapter}{Table of Contents}
%%------------List of Tables----------------------
\listoftables{}
\addcontentsline{toc}{chapter}{List of Tables}
%%------------List of Figures----------------------
\listoffigures{}
\addcontentsline{toc}{chapter}{List of Figures}

%change single space to double space
\doublespacing
%maintain Roman numerals on the previous page
\clearpage
%set pagination to Arabic
\pagenumbering{arabic} 



%%-----------Chapter start-------------------------------------
%%-----------Chapter 1------------------------------------------
\chapter{Referencing and Formatting}
\setcounter{secnumdepth}{3} \pagenumbering{arabic}
\setcounter{page}{1} \pagestyle{myheadings}
\markboth{}{}\markright{} \rhead{\thepage} \setcounter{page}{1}
\pagestyle{myheadings} \pagenumbering{arabic} \rhead{\thepage}
\setcounter{page}{1}

This chapter contains sample code for referencing and formatting.

\section{Footnotes, Endnotes, Bibliographic Citations and Cross-References}
Please see the relevant section below.

\subsection{Footnotes}
The following command is used for footnotes \textbackslash footnote\{text for the footnote\}.\footnote{text for the footnote}

\subsection{Endnotes}
If you wish to use endnotes instead of footnotes, remove the percent sign \% in front of the following command \textbackslash let\textbackslash footnote=\textbackslash endnote in the preamble of the MUNThesisBibTeX.tex file.

You can, then, use the footnote command as though it were an endnote command, i.e. \textbackslash footnote\{text for the endnote\}. At the end of the chapter, you enter the following command \textbackslash \{theendnotes\} to show the endnotes.


\subsection{Bibliographic citations}

The following command is used for bibliographic citations: \~{}\textbackslash cite\{UniqueBibliographyLabel\}.\cite{lshort}

\subsection{Cross-References}

The following command is used for cross-references: \textbackslash ref\{UniqueLabel\}. You can also indicate the page number with the following command: \textbackslash pageref\{UniqueLabel\}.\\

e.g. Please see the multiline textbox without a border (reference no. \ref{MultiNoBorder}) on page \pageref{MultiNoBorder}.\\

Note that the cross-reference command only works for labels (attached to a word or float) within the \emph{same} .tex file.

\section{Textboxes, Quotes, etc. and URIs}

This section covers textboxes, quotes, quotations, verses, URIs and e-mail addresses.

\subsection{Textboxes}
Please see the relevant subsubsection below for textbox options.

\subsubsection{Single-line textbox}

This set of commands produces a single-line textbox with a border width set at .125cm.\medskip

\textbackslash setlength\{\textbackslash fboxrule\}\{.125cm\}\\
\textbackslash fbox\{A single line of text: max. 66 characters\}

\medskip

\setlength{\fboxrule}{.125cm}
\fbox{A single line of text: max. 66 characters}

\subsubsection{Multi-line textbox with a border}

This set of commands produces a multi-line textbox with a border width set at .075cm.\medskip

\textbackslash setlength\{\textbackslash fboxrule\}\{.075cm\}\\
\textbackslash fbox\{\textbackslash parbox\{5cm\}\{Multiline text\}\}

\medskip

\setlength{\fboxrule}{.075cm}
\fbox{\parbox{5cm}
{Multiline text Multiline text Multiline text Multiline text Multiline text Multiline text Multiline text Multiline text Multiline text Multiline text Multiline text Multiline text Multiline text Multiline text Multiline text}}

\subsubsection{Multi-line textbox without a border}\label{MultiNoBorder}

This set of commands produces a centred, multi-line textbox without a border.\medskip

\textbackslash begin\{center\}\\
\{\textbackslash parbox\{5cm\}\{Multiline text\}\}\\
\textbackslash end\{center\}

\medskip

\begin{center} %Note American spelling
{\parbox{5cm}
{Multiline text Multiline text Multiline text Multiline text Multiline text Multiline text Multiline text Multiline text Multiline text Multiline text Multiline text Multiline text Multiline text Multiline text Multiline text}}
\end{center}

\subsection{Quotes, etc.}

The following subsubsections show quotes, quotations and verses.

\subsubsection{Quotes}

The following set of commands is used for quotes (the maximum length is sixty-six [66] characters):

\textbackslash begin\{quote\}\\
Quoted Text\\
\textbackslash end\{quote\}\\

This is how the quote appears \begin{quote}Quoted text\end{quote}

\subsubsection{Quotations}
For text longer than sixty-six (66) characters, you use the quotation set of commands:

\textbackslash begin\{quotation\}\\
Quoted Text\\
\textbackslash end\{quotation\}\\

This is how the quotation appears\begin{quotation}Quoted text Quoted text Quoted text Quoted text Quoted text Quoted text Quoted text Quoted text Quoted text Quoted text Quoted text Quoted text Quoted text Quoted text Quoted text\end{quotation}

\subsubsection{Verses}

The following set of commands is used for verse(s).To have a blank space between verses, you use the verse command for each verse.

\medskip

\textbackslash begin\{verse\}\\
Line 1 of the verse\textbackslash\textbackslash\\
Line 2 of the verse\textbackslash\textbackslash\\
Line 3 of the verse\textbackslash\textbackslash\\
Line 4 of the verse\textbackslash\textbackslash\\
\textbackslash end\{verse\}\\

For example:

\begin{verse}
Line 1 of the verse\\
Line 2 of the verse\\
Line 3 of the verse\\
Line 4 of the verse\\
\end{verse}

You can give the attribution with the \textbackslash hfill command after the double backslash \textbackslash\textbackslash  of the final line of verse and before the \textbackslash end\{verse\} command. You can also change the font shape, e.g. italic, bold.

\textbackslash hfil\textbackslash emph\{\textbackslash textbf\{Attribution\}\}

This command example inserts the attribution in italics and bold. Please note that the \textbackslash textbf command is enclosed in curly brackets \{\}.

\begin{verse}
Line 1 of the verse\\
Line 2 of the verse\\
Line 3 of the verse\\
Line 4 of the verse\\\hfill\textit{\textbf{Attribution}}
\end{verse}

\subsection{URIs and eMail}
The \textbackslash url\{\} command is used to properly show an URI or e-mail address:

School of Graduate Studies: \url{www.mun.ca/sgs}\\
eMail: \url{eTheses@mun.ca}


%%----------Chapter 2------------------------------------------
\chapter{Tables, Figures and Lists}

This chapter contains sample code for tables, figures and lists.

\section{Tables}
\subsection{Tables with a border}
Here is a table with a border:

\begin{table}[ht]
\caption{Sample Table With a Border}
\label{TableBorder}
\begin{center} %Note American spelling

%the “c” specifies the alignment of the text within the table cell (l=left, r=right, c=centre); brokenbars, i.e. |, insert vertical borders between the cells. 

\begin{tabular}{| c | c | c | c | c | c |}

%The \hline command inserts a horizontal line

\hline

%the ampersand & indicates next cell and the double backslash \\ indicates the end of the row

Row1 Col1 &Row1 Col2 &Row1 Col3 &Row1 Col4 &Row1 Col5 &Row1 Col6 \\
\hline
Row2 Col1 &Row2 Col2 &Row2 Col3 &Row2 Col4 &Row2 Col5 &Row2 Col6 \\
\hline
Row3 Col1 &Row3 Col2 &Row3 Col3 &Row3 Col4 &Row3 Col5 &Row3 Col6 \\
\hline
Row4 Col1 &Row4 Col2 &Row4 Col3 &Row4 Col4 &Row4 Col5 &Row4 Col6 \\
\hline

\end{tabular}
\end{center}
\end{table}

\subsection{Tables without a border}
Here is a table without a border:

\begin{table}[ht]
\caption{Sample Table Without a Border}
\label{TableNoBorder}
\begin{center} %Note American spelling
%the “c” specifies the alignment of the text within the table cell (l=left, r=right, c=centre).
\begin{tabular}{c c c c c c}

%the ampersand & indicates next cell and the double backslash \\ indicates the end of the row

Row1 Col1 &Row1 Col2 &Row1 Col3 &Row1 Col4 &Row1 Col5 &Row1 Col6 \\
Row2 Col1 &Row2 Col2 &Row2 Col3 &Row2 Col4 &Row2 Col5 &Row2 Col6 \\
Row3 Col1 &Row3 Col2 &Row3 Col3 &Row3 Col4 &Row3 Col5 &Row3 Col6 \\
Row4 Col1 &Row4 Col2 &Row4 Col3 &Row4 Col4 &Row4 Col5 &Row4 Col6 \\

\end{tabular}
\end{center}
\end{table}

\subsection{Informal tables}
Here is an informal table:

\begin{center}%Note American spelling

%the “c” specifies the alignment of the text within the table cell (l=left, r=right, c=centre).

\begin{tabular}{c c c c c c}

Row1 Col1 &Row1 Col2 &Row1 Col3 &Row1 Col4 &Row1 Col5 &Row1 Col6 \\
Row2 Col1 &Row2 Col2 &Row2 Col3 &Row2 Col4 &Row2 Col5 &Row2 Col6 \\
Row3 Col1 &Row3 Col2 &Row3 Col3 &Row3 Col4 &Row3 Col5 &Row3 Col6 \\
Row4 Col1 &Row4 Col2 &Row4 Col3 &Row4 Col4 &Row4 Col5 &Row4 Col6 \\

\end{tabular}
\end{center}

\section{Figures}
Here are the commands for inserting and manipulating figures. All figures use the following set of commands:\medskip

\textbackslash begin\{figure\}\\
\textbackslash includegraphics\{FigureFileNameWithoutExtension\}\\
\textbackslash caption\{Figure caption text\}\\
\textbackslash label\{UniqueNameToUseForReferencing\}\\
\textbackslash end\{figure\}\\

\subsection{Including an unformatted figure}
The set of codes above without any arguments and with the addition of \textbackslash begin\{center\} and \textbackslash end\{center\}.\medskip

See figure \ref{MUNFullSize} on page \pageref{MUNFullSize}.\medskip

\begin{figure}
\begin{center}
\includegraphics{MUN}
\caption{MUN Students - unformatted}
\label{MUNFullSize}
\end{center}
\end{figure}

\subsection{Scaling a figure}
In the \textbackslash includegraphics command, a scale argument can be used to resize the figure.\medskip

\textbackslash includegraphics[scale=x]\{FigureFileNameWithoutExtension\}\medskip

See figure \ref{MUNScaled} on page \pageref{MUNScaled}.\medskip

\begin{figure}
\begin{center}
\includegraphics[scale=.25]{MUN}
\caption{MUN Students - scaled at 25\%}
\label{MUNScaled}
\end{center}
\end{figure}

\subsection{Rotating a figure}
In the \textbackslash includegraphics command, an angle argument can be used to rotate the figure.\medskip

\textbackslash includegraphics[angle=x]\{FigureFileNameWithoutExtension\}\medskip

See figure \ref{MUNRotated} on page \pageref{MUNRotated}.\medskip

\begin{figure}
\begin{center}
\includegraphics[scale=.25, angle=-45]{MUN}
\caption{MUN Students - scaled and rotated}
\label{MUNRotated}
\end{center}
\end{figure}

\subsection{Mirroring a figure}
The \textbackslash reflectbox command is used with the \textbackslash includegraphics command to mirror a figure.\medskip

\textbackslash reflectbox\{\textbackslash includegraphics\{FigureFileNameWithoutExtension\}\}\medskip

Note that the \textbackslash includegraphics command is contained within curly brackets \{\}.\medskip

See figure \ref{MUNMirror} on page \pageref{MUNMirror}.\medskip

\begin{figure}
\begin{center}
\reflectbox{\includegraphics[scale=.25]{MUN}}
\caption{MUN Students - scaled and mirrored}
\label{MUNMirror}
\end{center}
\end{figure}


\section{Lists}
\subsection{Bulleted lists}

Bulleted lists use the \textbackslash begin\{itemize\} and \textbackslash end\{itemize\} set of commands. Each list entry is set with the \textbackslash item command.

Here's an example of a bulleted list:

\smallskip
\textbf{List heading text}
\begin{itemize}
\item List Item 1
\item List Item 2
\item List Item 3
\end{itemize}

\subsection{Numbered lists}

Numbered lists use the \textbackslash begin\{enumerate\} and \textbackslash end\{enumerate\} set of commands. Each list entry is set with the \textbackslash item command.

Here's an example of a numbered list:

\smallskip
\textbf{List heading text}
\begin{enumerate}
\item List Item 1
\item List Item 2
\item List Item 3
\end{enumerate}

\subsection{Inline lists}

Inline lists use the \textbackslash begin\{inparaenum\} and \textbackslash end\{inparaenum\} set of commands. Each list entry is set with the \textbackslash item command.

Here's an example of an inline list:

Some paragraph text before \begin{inparaenum}[(i)]\item text for the first item, \item text for the second item, \item text for the third item etc. \end{inparaenum} and optionally more paragraph text.

\subsection{Description lists}

Description lists use the \textbackslash begin\{description\} and \textbackslash end\{description\} set of commands. Each list entry is set with the \textbackslash item[Item to be described] command.

Here's an example of a description list:

\begin{description}
\item[First item to be described] text for the first item
\item[Second item to be described] text for the second item
\item[Third item to be described] text for the third item
\end{description}

\subsection{Nested lists}

If you wish to nest one or more lists within a list (LaTeX default is set at a maximum of five [5] levels), you enclose the entire set of list commands to be nested with curly brackets \{\}.

Here's an example of a nested list:

%Start the main list
\begin{enumerate}
\item text for the first item in level 1
%Start the nested list
{\begin{enumerate}
\item text for the first item in level 2
\item text for the second item in level 2
\item text for the third item in level 2
%End the nested list
\end{enumerate}}
\item text for the second item in level 1
\item text for the third item in level 1
%End the main list
\end{enumerate}


%%-----------------Chapter 3---------------------------
\chapter{Mathematical Equations}

For a solid overview of creating equations, refer to Chapter 3 “Typesetting Mathematical Formulae” (p. 49ff) in \emph{The Not So Short Introduction to \LaTeXe} (lshort.pdf) which should be in the /latex/doc folder of your TeX installation.

\section{Simple equations}
Simple equations can equally be done with either of the following sets of commands:

\textbackslash (equation\textbackslash )\\
\$equation\$\\
\textbackslash begin\{math\}equation\textbackslash end\{math\}\\

Example 1: This is an example of an equation \(a^2 + b^2 = c^2\) contained in the text.

Example 2: This is an example of an equation $a^2 + b^2 = c^2$ contained in the text.

Example 3: This is an example of an equation
\begin{math}
 a^2 + b^2 = c^2
\end{math}
contained in the text.

\section{Offset non-numbered equations}

To offset the equation with centring, the parentheses () in the first example above are replaced with square brackets []. Here is an example of a non-numbered equation which is offset from the text and centred.

\[
\lim_{x \to a}f(x)
\]
\[
\left|\sum_{i=1}^n a_ib_i\right| \le \left(\sum_{i=1}^n
a_i^2\right)^{1/2} \left(\sum_{i=1}^n b_i^2\right)^{1/2}
\]

\section{Numbered equations}
Equations can also be numbered.

\subsection{Single-line numbered equations}

Single-line numbered equations use the \textbackslash begin\{equation\} and \textbackslash end\{equation\} set of commands.

Here is an example of a single-line numbered equation:
\begin{equation}
(a+b)^3 = (a+b)^2(a+b)
\end{equation}

\subsection{Multiline numbered equations}

Multiline numbered equations use the \textbackslash begin\{align\} and \textbackslash end\{align\} set of commands.

Here is an example of a multiline numbered equation:
\begin{align}
(a+b)^3 &= (a+b)^2(a+b)\\
&=(a^2+2ab+b^2)(a+b)\\
&=(a^3+2a^2b+ab^2) + (a^2b+2ab^2+b^3)\\
&=a^3+3a^2b+3ab^2+b^3
\end{align}

%%-----------Chapter 4------------------------------------------
\chapter{Non-English Text}

A wide-range of languages have been enabled in the template. Please note that the template is designed for written English with insertion of non-English text and/or characters; it does not include the grammar rules, hyphenation rules etc. for languages other than English. 

Since some specialized fonts may not be in your TeX installation, the custom commands given below have been commented out with the percent sign \% in the MUNThesis LaTeX template. Simply remove the percent signs \% to see the output on compile.

\section{Symbols}
For a comprehensive list of symbols, please see pp. 62-69 in \emph{The Not So Short Introduction to \LaTeXe} (lshort.pdf) which should be in the /latex/doc folder of your TeX installation.

\section{Latin-alphabet based languages}
You can type directly using accented Latin-alphabet characters without having to use the more traditional format, e.g. ê instead of \textbackslash\^{}e. 

\section{Non-Latin-alphabet based languages}
 Please see the relevant subsection below.

\subsection{Arabic script}
Since there is a conflict with the Devanagari package, you need to remove the percent sign \% in front of the command to load the ArabTeX package in the preamble of the MUNThesisBibTeX.tex file:

\%package conflicts - only one may be activated\\
\%\textbackslash usepackage\{arabtex\}\\
\%\textbackslash usepackage\{devanagari\}\\

You use the following command with ASCII coding to display text in Arabic script: \textbackslash RL\{ASCII coding\}, e.g.
%\RL{al-salAm _Alykm}

Please refer to the ASCII coding chart on p. 19 of the extended ArabTeX manual at \url{http://129.69.218.213/arabtex/html/arabdoc.pdf}.

\subsection{CJK - Chinese, Japanese and Korean}
\fontencoding{T2A}
\selectfont

\subsubsection{Simplified Chinese}
The custom \textbackslash zhs\{Simplified Chinese characters\} command is used to insert Simplified Chinese characters in the document. Since the first instance of the command does not print any characters, an empty command is inserted followed by the command to insert the desired Simplified Chinese characters, e.g.\\

 \textbackslash zhs\{\}\\
 \textbackslash zhs\{Simplified Chinese characters\}\\

%\zhs{}
%\zhs{中华人民共和国中国}

\subsubsection{Traditional Chinese}
The custom \textbackslash zht\{Traditional Chinese characters\} command is used to insert Traditional Chinese characters in the document. Since the first instance of the command does not print any characters, an empty command is inserted followed by the command to insert the desired Traditional Chinese characters, e.g.\\

 \textbackslash zht\{\}\\
 \textbackslash zht\{Traditional Chinese characters\}\\

%\zht{}
%\zht{台灣共和國}\\
%\zht{香港}\\
%\zht{你好}

\subsubsection{Japanese Kanji, Hiragana and Katakana}
You will need to have the IPA TrueType Mincho (Unicode) font installed in your TeX distribution. The custom \textbackslash nippon\{Kanji/Hiragana/Katakana characters in Unicode\} command is used to insert Japanese characters in the document. Since the first instance of the command does not print any characters, an empty command is inserted followed by the command to insert the desired characters, e.g.\\

\textbackslash nippon\{\}\\
\textbackslash nippon\{Kanji/Hiragana/Katakana characters in Unicode\}\\

%Kanji: \nippon{日本国}\\ 
%Hiragana: \nippon{こんにちは}\\
%Katakana:  \nippon{コンニチワ}\\

\subsubsection{Korean Hangul}
Due to the sheer complexity in correctly installing Korean fonts in LaTeX as well as combining jamos to create syllabic blocks, by far the easiest route is to use XeLaTeX to output a pdf file on compile (instead of, for example, pdfLaTeX). The principal advantage of using XeLaTeX for compilation is that you do not have to install any fonts in LaTeX; XeLaTeX will use the fonts already installed on your system.

In order to enter Hangul text, you need to remove the percent sign \% in front of the Hangul newfontfamily command in the CJK section of the preamble of the MUNThesisBibTeX.tex file:

{\setlength{\baselineskip}
{.5\baselineskip}
\%CJK commands\\
\%Simplified Chinese\\
...\\
\%Traditional Chinese\\
...\\
\%Japanese Kanji, Hiragana and Katakana\\
...\\
\%Korean Hangul\\
\%\textbackslash newfontfamily\textbackslash hangul\{Font to use for Hangul\}\\
\%Korean Hanja\\
...\\\par}

In addition, you need to specify which font type you wish to use keeping in mind that that the font you specify must already be installed on your machine, e.g. Batang, Dotum, Myungjo.

To insert Hangul text, use the following command:

\{\textbackslash hangul Hangul syllabic blocks\}

%{\hangul Hangul syllabic blocks}


\subsubsection {Korean Hanja}
The custom \textbackslash hanja\{Korean Hanja characters\} command is used to insert Korean Hanja characters in the document. Since the first instance of the command does not print any characters, an empty command is inserted followed by the command to insert the desired Korean Hanja characters, e.g.\\

\textbackslash hanja\{\}\\
\textbackslash hanja\{Korean Hanja characters\}\\

%\hanja{}
%\hanja{京釜}


\subsection{Cyrillic}
You can type directly in Cyrillic as you would in any of the Latin-alphabet based languages.\medskip

Россия, Україна, Беларусь, България, Македонија, Србија, Молдова\medskip

Most of the additional letters from non-Slavic languages written in Cyrillic are supported, e.g. Ӝ/ӝ, Ӟ/ӟ, Ӥ/ӥ, Ӵ/ӵ in Udmurt.

\subsection{Devanagari}

Since there is a conflict with the ArabTeX package, you need to remove the \% percent sign in front of the command to load the Devanagari package in the preamble of the MUNThesisBibTeX.tex file:

\%package conflicts - only one may be activated\\
\%\textbackslash usepackage\{arabtex\}\\
\%\textbackslash usepackage\{devanagari\}\\

Please note that text written in Devanagari must be pre-processed - see the LaTeX manual of the MUN LaTeX thesis template package for the steps for preprocessing. 

Wherever you wish to enter Devanagari script, you use the following command:

\{\textbackslash dn Preprocessed Devanagari script\}

%{\dn d\?vnAgrF}

\subsection{Greek}
You use the following command to type directly in Greek: \textbackslash\{textgreek\}, e.g. \textgreek{Ελλάδα}

\subsection{Hebrew}
You use the following command with ASCII coding to display text in Hebrew:  \textbackslash cjRL\{ASCII coding\}, e.g. %\cjRL{/slwm `lykm} 

Please refer to the ASCII coding chart on p. 3 of the CJHebrew manual in the /latex/doc folder of your TeX installation.

%ensure correct pagination for the bibliography in the table of contents
\cleardoublepage

%add the bibliography to the Table of Contents
\addcontentsline{toc}{chapter}{Bibliography}

%add the bibliographic entries referenced in the .bib file
\bibliographystyle{plain}
\bibliography{MUNThesisBibTeX}
 
\end{document}
