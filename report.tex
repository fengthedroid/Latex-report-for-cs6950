\documentclass[12pt]{cls}
%set MUN Thesis Guidelines margins
\usepackage[left=3.8cm, right=2.5cm, top=3cm, bottom=3cm]{geometry}

%lists package and bullet formats
\usepackage{paralist}
\usepackage{bbding}
%enhanced graphics support
\usepackage{graphicx}
%formatting for websites and e-mail addresses
\usepackage{url}
%pageheaders and footers in LaTeX2e
\usepackage{fancyhdr}
%footnote options
\usepackage{footmisc}
%algorithms
\usepackage{algorithmic}
%source code printer
\usepackage{listings}
%read/write verbatim TeX code
\usepackage{fancyvrb}

\setlength{\parskip}{16pt plus 1pt minus 1pt}



%THIS IS WHERE YOU ENTER THE TITLE OF YOUR THESIS
\title{Class explorer for semantic web datastore}

%THIS IS WHERE YOU ENTER YOUR NAME
\author{Feng Wu}

%THIS IS WHERE YOU ENTER THE NAME OF YOUR DEGREE
\deg{Master of computer science}

%THIS IS WHERE YOU ENTER THE NAME OF YOUR DEPARTMENT, SCHOOL, or FACULTY
\fac{Department of Computer Science}

%THIS IS WHERE YOU ENTER THE DATE YOU SUBMITTED YOUR THESIS OR DISSERTATION
\date{June 2013}

%THIS IS WHERE YOU ENTER THE YEAR OF GRADUATION 
\copyrightyear{2013}


%no paragraph indentation
\setlength\parindent{0pt}
\newtheorem{theorem}{Theorem}[section]
\newtheorem{definition}{Definition}[section]
\newtheorem{lemma}{Lemma}[section]
\newtheorem{notation}{Notation}[section]
\begin{document}
\muntitlepage

%set the hierarchical drilldown to 3
\setcounter{secnumdepth}{3} \setcounter{tocdepth}{3}

%set pagination to Roman numerals and begin at page i
\pagenumbering{roman} \setcounter{page}{1}

\doublespacing
\setlength{\topmargin}{-.5in}

\chapter*{Abstract}
\addcontentsline{toc}{chapter}{Abstract}
%enter text for the abstract below
Text for the abstract.

%%-----------Table of Contents------------------
\renewcommand{\contentsname}{Table of Contents}
\tableofcontents{}
\addcontentsline{toc}{chapter}{Table of Contents}
%%------------List of Tables----------------------
\listoftables{}
\addcontentsline{toc}{chapter}{List of Tables}
%%------------List of Figures----------------------
\listoffigures{}
\addcontentsline{toc}{chapter}{List of Figures}

%change single space to double space
\doublespacing
%maintain Roman numerals on the previous page
\clearpage
%set pagination to Arabic
\pagenumbering{arabic} 



%%-----------Chapter start-------------------------------------
%%-----------Chapter 1------------------------------------------
\chapter{Introduction}
\setcounter{secnumdepth}{3} \pagenumbering{arabic}
\setcounter{page}{1} \pagestyle{myheadings}
\markboth{}{}\markright{} \rhead{\thepage} \setcounter{page}{1}
\pagestyle{myheadings} \pagenumbering{arabic} \rhead{\thepage}
\setcounter{page}{1}

\section{Background}

While the internet has successfully connected a huge amount of unstructured data in a human-orientated fashion, and hence produced structured data in a magnitude we have never seen before, ironically these structured data has been stored in completely separate silos. Numerous methods have been proposed to tear down the silo, both internally and externally, from the perspective of information enterprises. Tools such as middlewares and APIs are build upon existing data to address the issue.

On the other hand, with more relevant data generating exponentially every day, the state-of-the-art database design model - relational model - has been proven lacking the desired flexibility to adopt new information in many cases. In healthcare for instance, if the schema for patient records is predetermined, it will not be able to record the results for some tests that are invented after the database has been implemented. Entity–attribute–value model can be implemented on top of relational data schema for the extra dynamics, but this approach sacrifices the original functionalities that relational databases provide.

Rooted from the concept of semantic network, which is one of the knowledge representations in the domain of artificial intelligence, Semantic web solves the above difficulties by taking a fundamental paradigm shift about how the data should be organized.

Comparable to Entity–attribute–value model, semantic web stores data in a triplestore, where data entities are arranged as triples, which take the form of subject-predicate-object.

class

infer

\chapter{Task performed}
\chapter{Tools}
\chapter{How-to}

\chapter{Summary}
\section{Lesson learned}
\section{Future research}

\chapter{Appendix}
\section{References}
\section{A brief summary of the papers}

\subsection{Footnotes}
The following command is used for footnotes \textbackslash footnote\{text for the footnote\}.\footnote{text for the footnote}

\subsection{URIs and eMail}
The \textbackslash url\{\} command is used to properly show an URI or e-mail address:

School of Graduate Studies: \url{www.mun.ca/sgs}\\
eMail: \url{eTheses@mun.ca}


%ensure correct pagination for the bibliography in the table of contents
\cleardoublepage

 
\end{document}
